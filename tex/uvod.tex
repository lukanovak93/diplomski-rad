Kada se govori o biologiji i medicini, dvije grane koje se bave proučavanjem života i načina funkcioniranja živih bića, svi se slažu da su izuzetno puno napredovale sa razvojem tehnologije. Ljudi razvijaju sve složenije i sve preciznije alate, strojeve i programe za proučavanje svih sustava u tijelu, od stanice do skupa organa koji rade zajedno, kao jedan, te skupa čine savršeno ugođeni sustav. No u jednom području tih znanosti ne napreduje se tom brzinom kao u ostalim područjima. O tom području i sustavu organa ne zna niti približno mnogo kao o ostalim organima i sustavima u tijelu. Znanstvenici govore da je to najsloženiji sustav, najjače super-računalo na svijetu. Centar tog sustava je mozak, organ koji može probaviti nevjerojatnu količinu informacija u stvarnom vremenu sa gotovo savršenom precoznosti. \par

Od početka "modernog doba čovječanstva", javlja se ideja umjetne inteligencije. Znanstvenici iz područja tehničkih znanosti i računarstva teže razvijanju nečeg takvog i prirodno se nameće ljudski mozak kao ideal, razina inteligencije koju se želi dostići. Pretpostavlja se da je najveći dio mozga, oko 2/3, posvećen vidu \citep{vision_percentage}. Iz tog razloga, veliki dio proučavanja i razvoja moderne umjetne inteligencije fokusirano je na vid. U nastavku rada dan je pregled najvažnijih algoritama i arhitektura mreža.