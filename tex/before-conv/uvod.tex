Povijest prepoznavanja (detekcije) objekata seže puno dalje od konvolucijskih neuronskih mreža. Prvi sustavi detekcije objekata problemu su prilazili geometrijski. Neki od takvih pristupa bili su:

\begin{itemize}
\item \textbf{poravnanje} gdje se pokušavalo pronaći takvu transformaciju slike koja bi minimizirala pogrešku:
\[\sum_{i} residual(T(x_i), x_i')\]

Ovaj period trajao je od ranih 60ih do početka 90ih godina prošlog stoljeća. Uglavnom se pokušavalo riješiti problem detekcije razlamanjem objekata na slici u manje komponente (\textit{engl. Block world}, L. G. Roberts Machine Perception of Three Dimensional Solids, 1963.).
\item \textbf{modeli temeljeni na izgledu} (\textit{engl. appearance - based models}) su modeli temeljeni na svojstvenim vrijednostima (Eigenfaces for Recognition, Turk and Pentland, 1991.) te histogramima boja (Swain and Ballard, IJCV 1991.)
\item \textbf{klizeći prozor} pristup koji se donekle i danas primjenjuje. Ovaj pristup smatra se početkom "modernog doba" računalnog vida, iako se klizeći prozor koristi u najjednostavnijijm mogućim oblicima. Ovaj period počinje ranih 90ih godina, a predvode ih Turk and Pentland svojim radom "Eigenfaces for Recognition" izdanim 1991. godine, a vrhunac se potiže 2001. godine objavljivanjem algoritma koji su razvili Viola i Jones.
\item \textbf{lokalne značajke} (\textit{engl. local features}) gdje su oblici predmeta koje je potrebno detektirati djelomično poznati (D. Lowe, 2004.). U tom periodu Google razvija i prvi pretraživač slika. 
\item \textbf{parts-and-shape modeli} koji koriste kombinaciju više vrsta značajki: ponovo razlamaju objekte na dijelove, relativne odnose između tih dijelova te samu prisutnost dijelova objekta (Weber, Welling and Perona, 2000.).
\item \textbf{bags-of-features modeli} uvode prepoznavanje tekstura. Klasični bag-of-features modeli imaju standardizirane korake (\textit{engl. pipeline}): 
\begin{enumerate}
\item Izvlačenje značajki
\item Učenje "vizualnog vokabulara"
\item Prevođenje slika pomoću "riječi" iz "vizualnog vokabulara"
\item Reprezentacija slika pomoću "riječi" iz "vizualnog vokabulara"
\end{enumerate}
\end{itemize}