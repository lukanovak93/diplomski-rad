\section{Računalni vid}
Računalni vid je veliko područje računarske znanosti koje se bavi algoritmima i metodama vezanim za obradu slike. Kao što je navedeno u uvodu, vid je osjetilo preko kojega ljudi primaju najviše informacija, više nego preko bilo kojeg drugog osjetila. Da bi bilo moguće napraviti stroj ili program koji vidi kao mi, prvo je potrebno shvatiti kako ljudi vide i kako obrađuju informacije primljene preko očiju. Samo shvaćanje kako funkcionira ljudski vid je izuzetno težak zadatak. Danas donekle razumijemo taj proces, no ima još puno stvari koje su ostale neodgovorene. Recimo, istraživanja pokazuju da ljudi mogu prepoznati što je na slici za samo 13 milisekundi\citep{RSVP13ms}. Ljudi i dalje ne znaju kako mozak to radi. Postoje nagađanja, no ništa nije dokazano i upitno je hoće li ikada ti koncepti biti do kraja shvaćeni i objašnjeni. Zbog složenosti ovog problema, nemoguće je naći riješenje klasičnim pristupom i napisati program eksplicitno. Zato se u ovakvim problemima pribjegava umjetnoj inteligenciji.
Postoji nekoliko dijelova cijelog sustava vida koji ljudima omogućavaju vid:
\begin{itemize}
\item{\textbf{Vidjeti}} - dati računalu vid znači dati mu oči. U tom području znanost je dosta napredovala tako da danas postoje kamere sa većom rezolucijom od ljudskog oka. Ovaj dio vida odvija se, naravno, u samom oku.
\item{\textbf{Prepoznati}} - moći prepoznati različite objekte, maknuti šum sa slike itd. Ukratko, vidjeti što se na slici nalazi. Ovaj aspekt vida odvija se unutar oka i na putu do mozga gdje mozak na kraju posloži sve te informacije u smislenu sliku.
\item{\textbf{Razumijeti}} - shvatiti što se na slici nalazi, shvatiti odnose među objektima, propoznati različite predmete iz različitih kuteva... Ljudski mozak sposoban je prepoznati jabuku bila ona crvena, žuta ili zelena. Sposoban je prepoznati jabuku i ako je napola pojedena. Sposoban je prepoznati jabuku neovisno o njenoj orjentaciji, neovisno o tome miruje li ili se kreće. Ovaj dio vida odvija se potpuno u mozgu. Ovdje je trenutno dosegnuta granica računalnog vida i umjetne inteligencije. No, to ne znači da je nemoguće ići dalje, nego samo da je znanost savladala dosadašnje probleme i trenutno se bavi ovim dok ne nađe riješenje. 
\end{itemize}
