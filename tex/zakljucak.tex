U zadnjih nekoliko godina računalni vid izuzetno je brzo napredovao. Prije konvolucijskih mreža javljali su se algoritmi kao što je onaj Viole i Jonesa koji i danas zadovoljavajuće obavljaju zadaću detekcije objekata. No, ne može se o takvim algoritmima govoriti kao o umjetnoj inteligenciji jer oni rade na razini piksela i temelje se na statističkim metodama, alatima i zaključcima. Pojavom kolvolucijskih neuronskih mreža, računalni vid doživio je iznenadan napredak u svim smjerovima: klasifikacija, lokalizacija, detekcija, segmentacija itd. No, za razvitak neuronskih mreža sposobnih za složenije i točnije obavljanje ovih zadataka. bilo je potrebno razviti i brže metode treniranja. One su stigle razvojem programskih alata kao što je NVidia CUDA, progrmaski jezik za grafičke kartice koje množenje matrica izvode ouno brže od procesora. Tako je od 2013 nadalje počeo rapidni razvoj moćnih algoritama i neuronskih mreža kao što je YOLO. Ovakvi algoritmi imaju svoje nedostatke, ali se oni, za razliku od recimo Viola-Jones algoritma, mogu smatrati umjetnom inteligencijom. YOLO, iako nevjerojatno brz i precizan, i dalje nije savršen. Dalji koraci u napretku ovog algoritma izvedenig u radu bili bi popravljanje praćenja trajektorija na način da se pokuša naći bolji način od IoU metode sa prijašnjim i trenutnim okvirom. Nadalje, bilo bi dobro isprobati različite transformacije slike. Ili čak vršenje detekcije uz joše jednu kameru, po mogućnosti termalnu. Na taj način bi algoritam imao informaciju o false-positivima koje je očitao sa klasične slike/videa. No iako vrlo moćan algoritam, ostaje pitanje hoće li ikada biti dovoljno sposoban da se izjednači sa ljudskim okom i živčanim sustavom? Ili se mora javiti neka nova paradigma, neki novi pristup računalnom vidu i umjetnoj inteligenciji općenito? 